% In your .tex file
% !TEX program = xelatex
  !TEX program = latex
  
  
\documentclass[letterpaper,spanish,12pt]{article}

\usepackage[spanish]{babel}
\usepackage[T1]{fontenc}
\usepackage[latin1]{inputenc}

\usepackage{latexsym}
\usepackage{amsmath}
\usepackage{amssymb}
\usepackage{fancyhdr}
\pagestyle{fancy}


%\usepackage{doc,spanish}

%\usepackage{astronsp} % astron style file for spanish laguage

\selectlanguage{spanish}

\hyphenation{pa-la-bras}

\begin{document}


\title{Informaci\'{o}n necesaria para leer y poder entender los
  art\'{\i}culos del \\ 
M\'{o}dulo Producci\'{o}n Primaria, \\  
Universidad Aut\'{o}noma Metropolitana \\
Unidad Xochimilco, CBS. \\
Depto. El Hombre y su Ambiente \\
Lic. en Biolog\'{\i}a}

\author{Profesor Jos\'{e} A. Ar\'evalo R.}

\date{Trimestre 14O}

\maketitle

\begin{abstract}

Este documento presenta el alfabeto griego y los prefijos m\'{a}s
utilizados del Sistema Internacional de unidades (SI).
\end{abstract}


\section{Alfabeto Griego y prefijos del SI de unidades}

Se presenta parte del alfabeto griego y los prefijos m\'{a}s
utilizados del Sistema Internacional de unidades, \textbf{SI}. Este
documento y sus secciones tienen como objetivo de servir de ayuda a
los estudiantes para reconocer los s\'{\i}mbolos (alfabeto) y unidades
presentes en los diferentes art\'{\i}culos que se leer\'{a}n durante
el trandscurso de trimestre Producci\'{o}n Primaria Terrestres. Es
recomendable tener una copia de esta informaci\'{o}n siempre
disponible para su consulta inmediata.


\subsection*{Alfabeto Griego}
%\vspace{0.5cm}
%\bigskip

\begin{center}
\begin{tabular}{ccccc}
 $\Gamma     $ Gama&
 $\Delta     $ Delta&
 $\Theta     $ Teta&
 $\Lambda    $ Lambda&
 $\Xi        $ Xi \\ 
 $\Pi        $ Pi&
 $\Sigma     $ Sigma&
 $\Upsilon   $ Epsilon&
 $\Phi       $ Fi&
 $\Psi       $ Psi \\
 $\Omega     $ Omega & 
 $\alpha     $ alpha &
 $\beta      $ beta&
 $\gamma     $ gama &
 $\delta     $ delta \\
 $\epsilon   $ epsilon &
 $\varepsilon$ epsilon&
 $\zeta      $ zeta&
 $\eta       $ eta&
 $\theta     $ teta \\
 $\vartheta  $ teta &
 $\iota      $ iota &
 $\kappa     $ kapa& 
 $\lambda    $ lambda&
 $\mu        $ mu \\
 $\nu        $ nu &
 $\xi        $ xi&
 $o          $ o&
 $\pi        $ pi&
 $\varpi     $ pi \\
 $\rho       $ rho&
 $\varrho    $ rho&
 $\sigma     $ sigma &
 $\varsigma  $ sigma&
 $\tau       $ tau \\
 $\upsilon   $ epsilon&
 $\phi       $ phi&
 $\varphi    $ phi&
 $\chi       $ chi&
 $\psi       $ psi \\
 $\omega     $ omega &
 $        $ &
 $        $ &
 $        $ &
 $        $ \\
\end{tabular}
\end{center}
\vspace{0.5cm}


\subsection*{Prefijos del sistema internacional de unidades, SI.}

\vspace{0.5cm}
\begin{center}
\begin{tabular}{lcc}
M\'{u}ltiplo & Prefijo & S\'{\i}mbolo \\ \hline
10$^{-3}$ & mili & m \\ 
10$^{-6}$ & micro & $\mu$ \\
10$^{-9}$ & nano & n \\  
10$^{-12}$ & pico & p \\ 
10$^{-15}$ & fento & f \\ 
10$^{-18}$ & ato & a \\ \hline \hline
10$^{3}$ & kilo & k \\ 
10$^{6}$ & mega & M\\   
10$^{9}$ & giga & G\\   
10$^{12}$ & tera & T \\   
10$^{15}$ & peta & P\\   
10$^{18}$ & exa & E \\ \hline
\end{tabular}
\end{center}
\vfill


Cualquier comentario o aclaraci\'{o}n sobre este texto, favor de
comunicarlo a Jos\'{e} Ar\'{e}valo al correo electr\'{o}nico
\textbf{jarevalo@correo.xoc.uam.mx}.

%%%%%%%%%%%%%%%%%%%%%%%%%%%
\end{document}
